% Copyright (c) 2008 Jérémie DECOCK

\documentclass[pdftex,a4paper,11pt]{article} 
\usepackage[utf8]{inputenc}
\usepackage[frenchb]{babel}
\usepackage{hyperref}

\hypersetup{
	pdftoolbar=true,                                          % show Acrobat’s toolbar ?
	pdfmenubar=true,                                          % show Acrobat’s menu ?
	pdffitwindow=true,                                        % page fit to window when opened
	pdftitle={Les wikis},                                     % title
	pdfauthor={Jérémie DECOCK},                               % author
	pdfsubject={Améliorer l'intégration et l'efficacité des SI grâce aux wikis}, % subject of the document
	pdfnewwindow=true,                                        % links in new window
	pdfkeywords={wiki, entreprise, intégration, SI},          % list of keywords
	colorlinks=true,                                          % false: boxed links; true: colored links
	linkcolor=black,                                          % color of internal links
	citecolor=black,                                          % color of links to bibliography
	filecolor=black,                                          % color of file links
	urlcolor=black                                            % color of external links
}

\begin{document}

\title{Les wikis\\\medskip
       Un levier pour améliorer l'intégration et l'efficacité des SI~?}
\author{Jérémie \bsc{Decock}}
\date{22 février 2008}

\maketitle

%%%%%%%%%%%%%%%%%%%%%%%%%%%%%%%%%%%%%%%%%%%%%%%%%%

\section*{Introduction}
Les wikis font partis des applications phares du Web 2.0\footnote{On appelle Web 2.0 les applications web qui permettent aux internautes d'interagir avec le contenu des pages ou d'interagir entre eux.}. Un wiki est un site internet que tout le monde a le droit de modifier. Ses lecteurs peuvent donc en enrichir le contenu. Aucune connaissance en informatique n'est nécessaire pour modifier les pages d'un tel site, un formulaire permet de modifier chaque page ou d'en créer de nouvelles. Pour éviter le vandalisme, les modifications sont mémorisées par le système. N'importe qui peut annuler une modification malveillante. C'est simple, rapide et très efficace.

Le wiki est une technologie relativement jeune\footnote{Le concept du wiki a été inventé en 1995 par Ward \bsc{Cunningham} mais il ne s'est démocratisé qu'à partir des années 2000.} mais elle a fait ses preuves grâce à des sites tels que Wikipédia\footnote{Wikipedia (\url{http://www.wikipedia.org/}) est un des plus grands succès du web. Il fait partie des dix sites les plus consultés au monde. Il fait également parti des principaux ambassadeurs de ce qu'on appelle \og{}l'intelligence collaborative\fg{}.}. Un tel succès incite de plus en plus d'entreprises à l'utiliser dans leur système d'information. On parle alors de \emph{Wiki Business}.

Mais est-ce réellement efficace ou sommes-nous face à un simple effet de mode~? Les wikis peuvent-ils vraiment améliorer l'intégration et l'efficacité des SI~?

\section{Les principaux avantages pour l'entreprise}
\subsection{La simplicité, la légèreté et le prix}
Les wikis sont très simples, très légers (comparé aux ERP) et bien souvent ils sont peu onéreux (voire gratuit). Ils sont donc à la portée de n'importe quelle entreprise.

\subsection{Plus de spontanéité pour plus d'informations}
Modifier une page sur wiki est si simple et si rapide\footnote{Deux clics suffisent : un pour éditer la page et un autre pour enregistrer les modifications.} que de nombreux utilisateurs l'emploient même pour de petites modifications (corriger une faute d'orthographe, etc.). Ils peuvent ajouter des informations de façon plus spontanée, ce qui évite à l'entreprise de perdre la partie la plus informelle de ses connaissances.

\subsection{Structurer la pensée et l'information}
Comme n'importe quel site web, les wikis sont constitués de pages reliées entre elles par des hyperliens. Ces liens permettent de structurer l'information beaucoup plus efficacement qu'avec un document texte classique. On peut ainsi lier et organiser des idées et des informations pour rendre le contenu des pages interactif.

Mais contrairement aux sites web, les pages qui constituent un wiki peuvent être modifiées rapidement et facilement. C'est donc un outil adapté pour centraliser et cartographier rapidement la pensée (on parle de \emph{mind mapping}).

Ce bloc note virtuel peut éventuellement être accessible depuis l'extérieur ce qui est pratique quand on est en déplacement.

\subsection{Favoriser la coopération et le travail d'équipe}
Le mode de fonctionnement centralisé et ouvert des wikis leur permet de stimuler le travail d'équipe, la coopération et l'échange entre employés. Il met en valeur les compétences de chacun et responsabilise les acteurs.

\subsection{Une plus grande réactivité}
Un wiki n'a pas de structure a priori. Son squelette se construit au fur et à mesure en ajoutant des pages et des liens. Cet agencement est libre et peut être modifié à tout moment. Un wiki est donc très malléable et cette souplesse permet une grande réactivité.

\subsection{Remplacer avantageusement de nombreux outils}
La souplesse offerte par les wikis permet de remplacer avantageusement de nombreux outils : gestion de projet, CRM, SRM, GED, \emph{knowledge management}, CMS, \dots

Ils peuvent également permettre d'effectuer de nombreuses petites tâches : organiser la vie de l'entreprise, préparer les réunions et leur compte-rendu, faire un \emph{brain storming}\dots

\subsubsection{La GED et Knowledge Management}
L'intelligence collaborative émanant du wiki permet souvent de produire une documentation plus étoffée et de meilleure qualité car chacun peut apporter son savoir.

De plus, comme elle est centralisée, l'information est plus facile à retrouver.

\subsubsection{Le suivi de projet}
La souplesse offerte par les wikis est particulièrement bien adaptée pour gérer des projets jeunes et innovants aux contours pas toujours bien définis. On pourra regrouper sur la page d'un projet sa documentation, les grandes lignes à suivre, les tâches, les responsables, le planning d'avancement, les dates des délivrables ou encore le suivi des bugs.

\subsubsection{Le suivi commercial}
Le wiki peut être utilisé par des commerciaux pour documenter leur relation avec des clients ou l'état des négociations en cours. Il semble judicieux de créer alors une page par client décrivant son statut, le nom du consultant chargé de l'affaire, les négociations en cours, \dots

\subsection{Améliorer la communication intra et inter-entreprise}
Un wiki peut remplacer avantageusement les systèmes de messagerie. Par exemple, les emails ont tendance à diluer l'information et reconstituer l'historique d'une conversation n'est pas toujours évident. Au contraire les wikis centralisent l'information et son historique.

La plupart des wikis permettent de suivre l'activité d'une page en s'abonnant à un flux RSS ou Atom. On peut ainsi rester informé des dernières évolutions d'une discussion.

\subsection{Améliorer l'intégration des SI}
Une entreprise peut créer dans son wiki un espace destiné à ses clients, à ses fournisseurs ou à ses partenaires. Utilisé conjointement avec les technologies de syndication de contenu (RSS ou Atom), le wiki peut améliorer l'intégration des systèmes d'information. Il peut également être hébergé chez un prestataire externe. La zone d'échange devenant indépendante des partenaires, le wiki peut alors répondre aux impératifs de confidentialité des données.

%%%%%%%%%%%%%%%%%%%%%%%%%%%%%%%%%%%%%%%%%%%%%%%%%%

\section{Les limites d'un tel outil}
\subsection{Une interface pas toujours très conviviale}
L'édition des pages est rarement intuitive sur un wiki. Contrairement aux logiciels de traitement de texte WYSIWYG\footnote{What You See Is What You Get : lors de l'édition, l'utilisateur voit directement à quoi ressemblera l'affichage final (exemple : word et open office writer).}, les wikis ne proposent généralement pas d'interface graphique permettant de composer visuellement la mise en page du texte.

Ainsi, à l'instar des pages HTML, il faut utiliser des balises ou des séquences de caractères pour créer des liens, ajouter des images mettre des caractères en gras ou en italique, etc. On parle de \emph{syntaxe wiki} pour définir l'ensemble de ces balises ou séquences.

\subsection{Une syntaxe qui n'est pas standardisée}
Malheureusement, la \emph{syntaxe wiki} n'est pas standardisée et peut varier d'un wiki à l'autre.

\subsection{Le manque d'interopérabilité}
Il est généralement difficile de migrer des données vers des wikis utilisant un moteur différent. Certains moteurs font plus d'efforts que d'autres en permettant  par exemple d'exporter les pages wiki sous forme XML.

\subsection{Un coût humain souvent décourageant}
Les wikis peuvent être très efficaces si ils sont bien utilisés. Dans le cas contraire, ils ne servent à rien. Ils sont si différents des outils traditionnels qu'une adaptation de la mentalité et des méthodes de travail des (futurs) utilisateurs est souvent (si ce n'est pour dire toujours) nécessaire. Un effort important doit donc être fourni pour mettre en place une bonne conduite du changement.

%%%%%%%%%%%%%%%%%%%%%%%%%%%%%%%%%%%%%%%%%%%%%%%%%%

\section*{Conclusion}
Nous avons vu que les wikis possèdent de nombreux atouts faisant penser qu'ils ont leur place dans les SI des entreprises. Le principal obstacle à leur adoption massive reste le changement des mentalités et des méthodes de travail qu'ils imposent. Une conduite du changement efficace est donc indispensable pour profiter de ces atouts.

Pour les grandes entreprises, les wikis permettent avant tout une circulation plus fluide et transversale de l'information. Ainsi, pour la mise en place de son ERP, Michelin China a choisi d'utiliser un wiki pour faciliter la circulation de l'information à la fois à l'intérieur de l'équipe projet, mais aussi avec les partenaires\footnote{\url{http://twiki.org/cgi-bin/view/Main/TwikiSuccessStoryOfMichelinChina}}.

Pour une entreprise de petite taille, le principal intérêt du wiki est de permettre une meilleure structuration de l'information et un meilleur suivi des activités.

\section*{Bibliographie}
\bsc{Delacroix} Jérôme, \emph{Les wikis}, M2 Éditions, 2005
\end{document}
