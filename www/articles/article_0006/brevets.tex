% Copyright (c) 2008 Jérémie DECOCK (http://www.jdhp.org)

% This document is provided under the terms of the "Creative Commons BY-SA" license.
% For more details, read "legalcode.html" enclosed file or "http://creativecommons.org/licenses/by-sa/2.0/fr/" web page.

\documentclass{beamer}
\usepackage[utf8]{inputenc}
\usepackage[frenchb]{babel}
\usepackage{hyperref}

\hypersetup{
	pdftoolbar=true,                                          % show Acrobat’s toolbar ?
	pdfmenubar=true,                                          % show Acrobat’s menu ?
	pdffitwindow=true,                                        % page fit to window when opened
	pdftitle={Les brevets logiciels},                         % title
	pdfauthor={Jérémie DECOCK},                               % author
	pdfsubject={Les brevets logiciels en France},             % subject of the document
	pdfnewwindow=true,                                        % links in new window
	pdfkeywords={brevets logiciels},                          % list of keywords
	colorlinks=true,                                          % false: boxed links; true: colored links
	linkcolor=black,                                          % color of internal links
	citecolor=black,                                          % color of links to bibliography
	filecolor=black,                                          % color of file links
	urlcolor=black                                            % color of external links
}

%\usetheme{Singapore}

\title{Les brevets logiciels}
\author{Jérémie \bsc{Decock}}
\institute{Université Paris Descartes}
\date{30 mars 2008}

\begin{document}

\begin{frame}
\titlepage
\end{frame}

%%%%%%%%%%%%%%%%%%%%%%%%%%%%%%%%%%%%%%%

\begin{frame}
\frametitle{Plan}
\tableofcontents
\end{frame}

%%%%%%%%%%%%%%%%%%%%%%%%%%%%%%%%%%%%%%%

\section{Les brevets}
\begin{frame}
\begin{center}
{\LARGE Les brevets}
\end{center}
\end{frame}

\begin{frame}
\frametitle{Les brevets : objectifs}
Protéger une invention
\begin{itemize}
	\item Un titre de propriété industrielle qui confère à son propriétaire un droit exclusif d’exploitation sur une invention
	\item Valable 20 ans
	\item Valable que sur un territoire déterminé (un pays unique, ou un groupe de pays)
\end{itemize}
~\\
Favoriser la recherche et l'innovation
\begin{itemize}
	\item En contrepartie du droit d'exploitation exclusif accordé à son auteur, l'invention doit être divulguée au public et à l'issue de la période de protection, elle tombe dans le domaine public
\end{itemize}
\end{frame}

\begin{frame}
\frametitle{La portée des brevets}
La portée des brevets :
\begin{itemize}
	\item Brevets nationaux : un seul pays
	\item Brevets régionaux : plusieurs pays
\end{itemize}
~\\
Les organismes gérant les brevets :
\begin{itemize}
	\item France : INPI
	\item Europe : OEB (EPO)
	\item États Unis : USPTO
	\item ONU : OMPI (WIPO)
\end{itemize}
\end{frame}

\begin{frame}
\frametitle{Les critères de brevetabilité (en Europe)}
\begin{itemize}
	\item L’invention doit être nouvelle (Article L.~611-11 du CPI)
	\item L’invention doit impliquer une activité inventive (Article L.~611-15 du CPI)
	\item L’invention doit être susceptible d’application industrielle (Article L.~611-15 du CPI)
\end{itemize}
\end{frame}

\begin{frame}
\frametitle{Les travaux non brevetables}
Ne sont pas considérés comme des inventions, en raison de leur caractère abstrait (Article L.~611-10 du CPI) :
\begin{itemize}
	\item les découvertes, théories scientifiques et méthodes mathématiques
	\item les plans, principes et méthodes dans l’exercice d’activités intellectuelles, ainsi que les programmes d’ordinateur
	\item les présentations d’informations
\end{itemize}
~\\
Sont également exclues de la brevetabilité, pour des raisons éthiques (Article L.~611-17 du CPI) :
\begin{itemize}
	\item les inventions contraires à l’ordre public et aux bonnes mœurs
	\item les inventions concernant les espèces animales ou végétales
	\item les méthodes de traitement chirurgical ou thérapeutique du corps humain ou animal et les méthodes de diagnostic
\end{itemize}
\end{frame}

\begin{frame}
\frametitle{Classification internationale des brevets (CIB)}
Qu'est-ce que c'est ?\\
Un système hiérarchique permettant de classifier les brevets\\
~\\
Objectif :\\
Faciliter les recherches sur les brevets connus à l'échelle mondiale
\end{frame}

%%%%%%%%%%%%%%%%%%%%%%%%%%%%%%%%%%%%%%%

\section{Les brevets logiciels}
\begin{frame}
\begin{center}
{\LARGE Les brevets logiciels}
\end{center}
\end{frame}

\begin{frame}
\frametitle{La situation actuelle (en Europe)}
Directive européenne du 14 mai 1991 :
\begin{itemize}
	\item les logiciels sont protégés par le droit d'auteur
	\item les logiciels ne peuvent pas être brevetés
\end{itemize}
\end{frame}

\begin{frame}
\frametitle{La situation actuelle : les éléments protégés}
Les éléments protégés :
\begin{itemize}
	\item L'architecture du programme
	\item Le code source
	\item Le code objet (code source compilé)
	\item Les éléments multimédia incorporés (son, texte, image, vidéo)
	\item Les écrans et modalités d'intéractivité (s'ils sont originaux)
	\item Le matériel de conception préparatoire : les ébauches, les maquettes, les dossiers d'analyses fonctionnelles, la documentation intégrée au logiciel, les prototypes
\end{itemize}
\end{frame}

\begin{frame}
\frametitle{La situation actuelle : les éléments non protégés}
Les éléments non protégés :
\begin{itemize}
	\item Les fonctionnalités
	\item Les algorithmes
	\item Les interfaces
	\item Les langages de programmation
\end{itemize}
\end{frame}

\begin{frame}
\frametitle{Le point de vue des pro-brevet}
Leur principal reproche :\\
Le droit d'auteur ne protège que la forme et non le concept créatif (algorithmes et fonctionnalités)\\
~\\
Ce qu'ils souhaitent :\\
Protéger le concept inventif des logiciels, et non plus seulement sa mise en forme, ce que permettraient les brevets logiciels
\end{frame}

\begin{frame}
\frametitle{Le point de vue des anti-brevet}
Leur position :\\
Les brevets logiciels risquent de créer un déséquilibre en faveur des grands éditeurs et de renforcer les monopoles au détriment des petits éditeurs et des clients\\
~\\
Leur argument :\\
Les petites entreprises, les développeurs indépendants et les développeurs de logiciels libres n'auront pas les moyens financiers de s'engager dans la course aux brevets
\end{frame}

\begin{frame}
\frametitle{Le point de vue des anti-brevet}
Ce qu'ils affirment :\\
Le droit d'auteur appliqué aux logiciels ne bloque pas l'innovation\\
~\\
Ce qu'ils dénoncent :\\
Les dérives que l'on peut observer dans les pays appliquant les brevets logiciels (brevets portant sur des idées triviales ou sur des concepts)
\end{frame}

\begin{frame}
\frametitle{Une tentative de légalisation des brevets logiciels en Europe}
Propositions de directives au niveau européen :
\begin{itemize}
	\item vote (en première lecture) au Parlement Européen le 24 septembre 2003 à Strasbourg
	\item adoption (en première lecture) en Conseil des Ministres le 7 mars 2005 de l'accord politique sur les brevets logiciels du 18 mai 2004 à Bruxelles
	\item rejet par le parlement lors du vote (en deuxième lecture) le 6 juillet 2005 à Strasbourg
\end{itemize}
\end{frame}

\begin{frame}
\frametitle{Conclusion}
\begin{itemize}
	\item Les brevets permettent de stimuler les innovations industrielles en protégeant les procédés techniques tout en les ouvrant à la concurrence
	\item Pour ce qui est des logiciels, le système du droit d'auteur semble bien mieux adapté car ce sont des oeuvres de l'esprit à mi-chemin entre les oeuvres littéraires et les démonstrations mathématiques
\end{itemize}
\end{frame}

\begin{frame}
\frametitle{Sources}
\begin{itemize}
	\item IRPI : \url{http://www.irpi.ccip.fr}
	\item INPI : \url{http://www.inpi.fr}
	\item L'APP : \url{http://app.legalis.net}
	\item Legifrance : \url{http://www.legifrance.gouv.fr}
	\item Wikipedia : \url{http://fr.wikipedia.org}
	\item OEB : \url{http://www.epo.org}
	\item APRIL : \url{http://www.april.org/groupes/brevets/}
	\item nosoftwarepatents.com : \url{www.nosoftwarepatents.com}
	\item endsoftwarepatents.org : \url{http://endsoftpatents.org}
	\item AFUL : \url{http://www.aful.org}
	\item brevets-logiciels.info : \url{http://brevets-logiciels.info}
	\item FFII : \url{http://eupat.ffii.org}
	\item Pierre Breese : \url{http://www.breese.fr}
\end{itemize}
\end{frame}

\begin{frame}
\begin{center}
    \href{http://creativecommons.org/licenses/by-sa/2.0/fr/}{\includegraphics[width=.40\linewidth]{images/cc_by_sa}}
\end{center}
\end{frame}

\end{document}
