% Copyright (c) 2006,2008 Jérémie DECOCK (http://www.jdhp.org)

% This document is provided under the terms of the "Creative Commons BY-SA" license.
% For more details, read "legalcode.html" enclosed file or "http://creativecommons.org/licenses/by-sa/2.0/fr/" web page.

\documentclass[pdftex,a4paper,11pt]{report} 
\usepackage[utf8]{inputenc}
\usepackage[frenchb]{babel}
\usepackage[pdftex]{graphicx}
\usepackage{hyperref}

\hypersetup{
	pdftoolbar=true,                                          % show Acrobat’s toolbar ?
	pdfmenubar=true,                                          % show Acrobat’s menu ?
	pdffitwindow=true,                                        % page fit to window when opened
	pdftitle={La protection du logiciel par le droit d'auteur}, % title
	pdfauthor={Jérémie DECOCK},                               % author
	pdfsubject={La protection du logiciel par le droit d'auteur}, % subject of the document
	pdfnewwindow=true,                                        % links in new window
	pdfkeywords={droit d'auteur, logiciel, brevet},           % list of keywords
	colorlinks=true,                                          % false: boxed links; true: colored links
	linkcolor=black,                                          % color of internal links
	citecolor=black,                                          % color of links to bibliography
	filecolor=black,                                          % color of file links
	urlcolor=black                                            % color of external links
}

\begin{document}

\title{La protection du logiciel par le droit d'auteur}
\author{
	Jérémie \bsc{Decock}
}
\date{4 avril 2006}

\maketitle

%---------
% Chapter
%---------
\chapter*{Introduction}
\paragraph{}
En France, depuis la loi du 3 juillet 1985, les logiciels sont protégés par le droit d'auteur, au même titre que n'importe quelle oeuvre littéraire ou artistique. Aujourd'hui, certaines personnes prétendent que le droit d'auteur n'est pas adapté pour assurer une bonne protection des logiciels et demandent une remise en cause de ce choix.

Quelles raisons les motivent~? Le droit d'auteur appliqué aux logiciels protège-t-il efficacement les intérêts de chacun~?

%\paragraph{}
Nous présenterons dans un premier temps le contexte historique qui a amené au choix du droit d'auteur pour protéger les logiciels. Nous étudierons ensuite les aménagements qui ont été nécessaires pour appliquer ce choix. Nous discuterons enfin de l'efficacité de ce dispositif selon différents points de vue.

%---------
% Chapter
%---------
\chapter{Pourquoi les logiciels sont protégés par le droit d'auteur~?}

% Section
\section{Le contexte}
Au début de l'industrie informatique, les progiciels\footnote{\og Produit logiciel commercialisé de façon autonome et en standard, ce qui suppose un certain degré de portabilité \fg{} (Jean-Benoît \bsc{Zimmermann})} n'existaient pas. Soit les logiciels étaient conçus pour un matériel précis et n'étaient vendus qu'avec ce matériel, soit ils étaient développés sur mesure pour un client. Ils étaient considérés comme une activité industrielle dans le premier cas et comme un service dans le second. De fait, la problématique de la protection de ces logiciels ne se posait pas étant donné leur statut.

Par la suite, le logiciel s'est détaché du matériel et est devenu universel (c'est à dire conçu pour un grand nombre de clients). C'est l'apparition des progiciels. Ces progiciels ont pris une grande importance sur le marché et dés lors, leur statut est devenu ambiguë~: ils ne peuvent plus être considérés comme une activité industrielle ou un service.

Cette ambiguïté leur fait perdre toute protection car la loi ne prévoit pas de statut particulier pour ce bien marchand d'un genre nouveau. Selon André \bsc{Lucas} (1987), \og~le programme, qui est au coeur du logiciel, se caractérise d'abord par son contenu comme un procédé permettant de tirer parti des ressources de la machine en vue d'un résultat déterminé. A ce titre, sa protection pose un problème de brevetabilité. Mais en même temps, il se présente en lui-même, apparemment au moins, comme une oeuvre de l'esprit susceptible de donner prise au droit d'auteur [\dots]. De là vient qu'il puisse a priori prétendre à la fois au bénéfice du brevet d'invention et à celui du droit d'auteur. De là vient aussi qu'il ne puisse trouver commodément sa place ni dans l'une ni dans l'autre de ces deux branches du droit de la propriété intellectuelle, ce qui fait toute la différence en la matière~\fg.

Fallait-il concevoir un cadre juridique spécifique pour un tel produit qui est à la fois technologie et expression~?

% Section
\section{La solution retenue}
A partir des années 70, plusieurs pays adaptent leurs textes pour répondre au problème de la reconnaissance et de la protection des logiciels (ndl~: dans la suite de cette exposé, nous utiliseront indifféremment les termes de logiciels et de progiciels pour désigner la même chose). La plupart de ces pays ont choisi d'intégrer les logiciels dans le système des droits d'auteurs.

En France, la jurisprudence a longtemps refusé d'assimiler les logiciels à des oeuvres couvertes par le droit d'auteur. Ce n'est qu'en novembre 1982 que la Cour d'Appel de Paris a accordé un "droit d'auteur" à un informaticien, sur un programme qu'il avait conçu. Dès lors, la jurisprudence ne parvenait plus à suivre une orientation cohérente, étant donné la contradiction des précédents arrêts. Il fallut attendre la loi du 3 Juillet 1985, pour qu'une position officielle éclaircisse la situation. La décision fut prise d'ajouter les logiciels "originaux" à la liste des oeuvres protégées par le droit d'auteur. Ils furent ainsi intégrés dans la première partie du Code de la Propriété Intellectuelle, parmi les dispositions relatives à la propriété littéraire et artistique.

L'idée de concevoir de toute pièce un cadre juridique pour les logiciels n'a pas été retenue, car la vitesse avec laquelle évolue l'informatique risquait de rendre les textes rapidement obsolètes.

Si la loi du 3 Juillet 1985 a clarifié la situation, elle est restée floue sur les notions d'originalité et de protection. Elle a été remplacée par la loi du 10 mai 1994, transposition de la directive européenne du 14 mai 1991, qui a instauré à l'échelon européen un régime juridique unifié pour les logiciels.

%---------
% Chapter
%---------
\chapter{Le droit d'auteur et les logiciels}

% Section
\section{Les droits moraux et les droits patrimoniaux}
De la même manière que le droit d'auteur "traditionnel", le droit d'auteur sur le logiciel se composent de droits moraux et de droits patrimoniaux.

\subsection{Les droits moraux}
Comme pour toutes autres oeuvres, les droits moraux de l'auteur d'un logiciel ne peuvent ni être vendus, ni cédés (mais ils sont transmissibles aux héritiers après la mort de l'auteur). Toutefois, ils sont réduits puisque l'auteur ne dispose que du droit au respect de son nom et au libre choix de la divulgation de son oeuvre (articles L.121-1 et L.121-2). Le droit de repentir et le droit à l'intégrité sont suspendus au profit de l'acquéreur (article L.121-7).

\subsection{Les droits patrimoniaux}
Les droits patrimoniaux permettent à son détenteur et à lui seul d'effectuer et d'autoriser la reproduction, la traduction, l'adaptation, l'arrangement et la mise sur le marché de son logiciel (article L.122-6).

Il existe cependant une exception aux restrictions précédentes~: l'utilisateur a le droit d'étudier le fonctionnement interne du logiciel (par décompilation ou reverse engineering) et de modifier celui-ci dans le but de corriger des erreurs, de le rendre accessible à son utilisateur ou interopérable avec un autre logiciel ou un autre système (article L.122-6-1).

Le titulaire des droits patrimoniaux peut aussi dicter les conditions d'utilisation du logiciel au travers d'un contrat~: la licence (sous réserve qu'elle respecte la loi). L'apposition d'une licence sur un logiciel n'est pas obligatoire, son absence n'ayant aucune conséquence sur le droit d'auteur du logiciel. Elle permet simplement de fixer quelques règles entre l'utilisateur et le propriétaire (par exemple, le nombre de machine sur lesquelles le logiciel peut être installé). A tout logiciel installé sur un ordinateur doit correspondre une licence légalement acquise.

La reproduction intégrale ou partielle d'un logiciel sans le consentement de son auteur ou de ses ayants droits est illégale (article L.122-4).

% Section
\section{Les éléments protégés et les modalités de protection}
La loi ne distingue pas clairement les éléments protégés des éléments non protégés. La jurisprudence permet néanmoins de faire cette distinction.

\subsection{Éléments protégés}
\begin{itemize}
\item L'architecture du programme
\item Le code source
\item Le code objet (résultat de la compilation du code source)
\item Les éléments multimédia incorporés (son, texte, image)
\item Les écrans et modalités d'intéractivité (s'ils sont originaux)
\item Le matériel de conception préparatoire~: les ébauches, les maquettes, les dossiers d'analyses fonctionnelles, la documentation de conception intégrée au logiciel, les prototypes.
\end{itemize}

\subsection{Éléments non protégés}
\begin{itemize}
\item Les fonctionnalités
\item Les algorithmes
\item Les interfaces
\item Les langages de programmation
\end{itemize}

\subsection{Les modalités de protection}
Comme pour le droit d'auteur "traditionnel", aucune modalité n'est requise pour protéger un logiciel~: tout travaux est automatiquement protégé dès lors qu'il est concrétisé (article L.111-2). On peut toutefois préciser que~:
\begin{itemize}
\item si l'apposition de la mention Copyright n'est pas obligatoire, elle est vivement recommandée dans le cas d'une exploitation du logiciel à l'étranger (dans le droit anglo-saxon, la mention du copyright est obligatoire pour prétendre au droit d'auteur) ;
\item le dépôt du logiciel n'est pas obligatoire mais peut présenter un intérêt pour constituer des preuves d'antériorité en cas de litige. L'APP\footnote{http://app.legalis.net} (Agence pour la Protection des Programmes) fait partie des organisations habilités à recevoir des dépôts de logiciels. On peut également effectuer un dépôt auprès d'un huissier ou plus simplement s'envoyer par courrier les éléments à protéger, la date du cachet de la poste faisant foi.
\end{itemize}

% Section
\section{Les titulaires du droit d'auteur}
Selon l'article L113-1, la qualité d'auteur appartient à celui qui a pris l'initiative de créer et de réaliser le logiciel.

Derrière cette simplicité, différentes conditions de réalisation peuvent être observées :
\begin{itemize}
\item lorsque le logiciel est créé par un seul auteur, il appartient à celui-ci ;
\item lorsque le logiciel est créé par plusieurs personnes physiques, il appartient à ses différents auteurs et constitue une oeuvre de collaboration (article L113-3) ;
\item lorsque le logiciel est réalisé par plusieurs personnes sous la direction d'une personne physique ou morale qui a prie l'initiative de le créer, il appartient à cette personne et constitue une oeuvre collective (article L113-5).
\end{itemize}

Dans le cas d'un logiciel développé dans une entreprise les droits patrimoniaux sont réservés à l'employeur (Art. L113-9). Quant aux droits moraux, ils restent acquis à l'(aux) auteur(s) salarié(s) (Art. L121-1). Il en est de même pour les logiciels créés par les agents de l'état, de collectivités publiques et des établissements à caractère administratif.

%---------
% Chapter
%---------
\chapter{Différents points de vue}
Bien qu' aujourd'hui le droit d'auteur semble être un outil efficace et équilibré, certains prétendent qu'il n'est plus suffisant et demandent une remise en cause du statut des logiciels et de leur système de protection.

% Section
\section{Le point de vue des pro-brevets}

\subsection{Leur principal reproche}
Selon les pro-brevets, le droit d'auteur seul ne suffit plus à protéger efficacement les intérêts des développeurs et de leurs employeurs. En effet, le droit d'auteur ne protège que la forme et non le concept créatif, lequel réside dans l’algorithme et les fonctionnalités. Or selon eux, il représente non seulement la majorité des investissements en recherche et développement mais aussi la principale valeur ajoutée. 

Le droit d'auteur empêche le plagiat du code source d'un logiciel mais il autorise la concurrence à copier le concept de ce logiciel et ses mécanismes internes. Ainsi, le droit d'auteur seul n'encouragerait pas les entreprises à faire des investissements importants en recherche et développement. Les inventions n'étant que partiellement protégées, ces investissements ne seraient pas rentables.

\subsection{Ce qu'ils souhaitent}
Les pro-brevets veulent donc protéger le concept inventif des logiciels, et non plus seulement sa mise en forme.

Les brevets permettent cette protection pour des procédés techniques et pour les programmes nécessaires à l'obtention de ces procédés (par exemple les programmes contrôlant l'ABS d'une voiture ou les instruments de navigation d'un avion). En revanche les fonctionnalités d'un logiciel en tant que tel et hors de tout procédé technique, ne peuvent pas être breveté (par exemple brevet protégeant le paiement en un clic du site Amazone.com valable aux États Unis mais pas en Europe). Les pro-brevets souhaiteraient donc lever cette interdiction et étendre leur champ de protection aux concepts créatifs des logiciels.

\subsection{Les autres reproches faits au droit d'auteur}
\begin{itemize}
\item Le droit d'auteur seul ne permet pas aux PME de se faire une place sur le marché et entretiennent les monopoles
\item Le droit d'auteur seul ne favorise pas l'innovation ce qui est mauvais pour le public
\item Le droit d'auteur seul ne permet pas de résister à la pression des entreprises américaines
\item Le droit d'auteur seul ne permet pas de protéger l’innovation européenne
\end{itemize}

% Section
\section{Le point de vue des opposants aux brevets logiciels}
\subsection{Leur position}
Selon les "anti-brevets", le système de droit d'auteur est le seul à offrir une bonne protection des logiciels et de leurs auteurs sans négliger les intérêts du public, et des concurrents.

Ils considèrent par contre que la généralisation des brevets aux logiciels risque de rompre cet équilibre et de renforcer les monopoles existant au détriment des intérêts du public, des petites entreprises, des développeurs indépendants et des logiciels libres.

\subsection{Leurs arguments}
Les opposants aux brevets logiciels contestent la plupart des arguments avans part leurs adversaires :
\begin{itemize}
\item ils considèrent que les petites entreprises, les développeurs indpendants et les développeurs de logiciels libres n'auront pas les moyens financiers de s'engager dans la course aux brevets. Selon eux, déposer un brevet reste relativement chèr et plus encore, les coûts indirects nécessaires pour vérifier constamment si leurs travaux ne sont pas protégés. Il n'auraient probablement pas les moyens non plus d'acquérir les droits d'utilisation d'une invention protgée. Ces prédictions renforceraient ainsi les monopoles existants~;
\item en s'appuyant sur les arguments précédents, ils affirment que contrairement aux brevets logiciels, le droit d'auteur ne bloque pas l'innovation. Ils prennent pour exemple le web qui ne serait pas ce qu'il est si le CERN l'avait breveté. Le droit d'auteur serait donc plus  proche des intérêts du public~;
\item ils rappellent aussi que les brevets sont néfastes pour le développement des logiciels libres, qui sont pourtant la seule chance pour l'industrie du logiciel européen de rattraper son retard sur les États Unis et de gagner son indépendance. Cet argument pèse lourd dans certains secteurs stratégiques tels que l'administration et la défense. Il touche également certaines entreprises exposées à l'espionnage industriel~;
\item quant à l'ampleur des investissements de recherche et développement avancée par certaines entreprises, ils considèrent qu'elle est dans bien des cas exagrée (le paiement en un clic du site marchand Amazone.com semble trivial)~;
\item enfin, les anti-brevet s'inquiètent surtout des dérives que l'on peut observer outre Atlantique (course aux brevets, attribution aveugle des brevets, affaire SCO\footnote{sur le site http://linuxfr.org/ taper sco dans le champ de recherche (en haut à droite) pour voire l'historique de l'affaire}, le paiement en un clic d'Amazone, brevet du cabinet d'avocat de McKool Smith portant sur l'affichage d'images 3D sur un écran d'ordinateur\footnote{http://www.nofrag.com/2004/nov/02/14693/}, etc.).
\end{itemize}

%---------
% Chapter
%---------
\chapter*{Conclusion}
Finalement, bien que certains se plaignent des limites du système du droit d'auteur, il semble particulièrement bien équilibré, et concilie au mieux les intérêts de chacun.

Ainsi, selon Pierre \bsc{Breese}, pourtant favorable à la brevetabilité des logiciels, \og~la situation  actuelle [est], somme toute satisfaisante car elle conduit, en ce qui concerne l'innovation en informatique, à des décisions judiciaires raisonnables et à des procédures d'examen des brevets dans le domaine des logiciels acceptables. Cette situation concilie les besoins des entreprises innovantes de protéger le retour sur investissement de leurs efforts d'innovation, tout en laissant la place à un marché de logiciels "génériques" distribués sous licence dites "libres"~\fg.

Le seul point négatif réside dans le déséquilibre de la concurrence entre les entreprises européennes et les entreprises américaines, ces dernières utilisant massivement les brevets pour monopoliser l'industrie du logiciel.

%---------
% Chapter
%---------
\tableofcontents

%---------
% Chapter
%---------
\clearpage

\begin{center}
    \href{http://creativecommons.org/licenses/by-sa/2.0/fr/}{\includegraphics[width=.30\linewidth]{images/cc_by_sa}}\\
	Creative Commons BY-SA
\end{center}

\end{document}
