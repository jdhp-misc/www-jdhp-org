% Copyright (c) 2008 Jérémie DECOCK (http://www.jdhp.org)

% This document is provided under the terms of the "Creative Commons BY-SA" license.
% For more details, read "legalcode.html" enclosed file or "http://creativecommons.org/licenses/by-sa/2.0/fr/" web page.

\documentclass[pdftex,a4paper,11pt]{article} 
\usepackage[utf8]{inputenc}
\usepackage[frenchb]{babel}
\usepackage[pdftex]{graphicx}
\usepackage{hyperref}

\hypersetup{
	pdftoolbar=true,                                          % show Acrobat’s toolbar ?
	pdfmenubar=true,                                          % show Acrobat’s menu ?
	pdffitwindow=true,                                        % page fit to window when opened
	pdftitle={Introduction à la correction orthographique avec Vim}, % title
	pdfauthor={Jérémie DECOCK},                               % author
	pdfsubject={Introduction à la correction orthographique avec Vim}, % subject of the document
	pdfnewwindow=true,                                        % links in new window
	pdfkeywords={vim, correction orthographique, orthographe, dictionnaire, spell}, % list of keywords
	colorlinks=true,                                          % false: boxed links; true: colored links
	linkcolor=black,                                          % color of internal links
	citecolor=black,                                          % color of links to bibliography
	filecolor=black,                                          % color of file links
	urlcolor=black                                            % color of external links
}

\begin{document}

\title{Introduction à la correction orthographique avec Vim}
\author{Jérémie \bsc{Decock}}
\date{10 février 2008}

\maketitle

%%%%%%%%%%%%%%%%%%%%%%%%%%%%%%%%%%%%%%%%%%%%%%%%%%

\section{Préambule}
Ce document explique comment installer et utiliser la correction orthographique pour la langue française dans l'éditeur Vim. Il a été écrit pour fonctionner sur n'importe quel système GNU/Linux et devrait fonctionner à l'identique sur les systèmes BSD (non testé).

À partir de maintenant, nous supposons que Vim est correctement installé et configuré.\\

La documentation de référence peut être consultée depuis Vim en tapant \og{}:help\fg{} ou sur le web à l'adresse suivante : \url{http://vimdoc.sourceforge.net/htmldoc/spell.html}.

%%%%%%%%%%%%%%%%%%%%%%%%%%%%%%%%%%%%%%%%%%%%%%%%%%

\section{Installer les fichiers dictionnaires}
Pour installer le dictionnaire français sur un système GNU/Linux récent, il faut copier les fichiers suivants dans le répertoire \textasciitilde{}/.vim/spell/ :
\begin{itemize}
	\item \url{http://ftp.vim.org/vim/runtime/spell/fr.utf-8.spl}
	\item \url{http://ftp.vim.org/vim/runtime/spell/fr.utf-8.sug}
\end{itemize}
~\\
Sur les anciens systèmes GNU/Linux, il est possible que le codage par défaut ne soit pas l'UTF-8 mais l'ISO 8859-1 ou l'ISO 8859-15. Dans ce cas, il faut télécharger les fichiers correspondant sur la page \url{http://ftp.vim.org/vim/runtime/spell/}.

%%%%%%%%%%%%%%%%%%%%%%%%%%%%%%%%%%%%%%%%%%%%%%%%%%

\section{Activer la correction orthographique}
Maintenant que les dictionnaires sont installés, on peut les utiliser dans Vim en tapant la commande suivante :

\begin{verbatim}
:setlocal spell spelllang=fr
\end{verbatim}

Les fautes d'orthographe apparaissent alors en rouge. Pour éviter de devoir retaper cette commande à chaque lancement de Vim, il vaut mieux l'inscrire dans le fichier \textasciitilde{}/.vimrc.

%%%%%%%%%%%%%%%%%%%%%%%%%%%%%%%%%%%%%%%%%%%%%%%%%%

\section{Utiliser le correcteur orthographique}
Voici les principales commandes à connaître :
\begin{itemize}
	\item Pour corriger un mot, il faut placer le curseur dessus et taper \og{}z=\fg{}. Une liste de propositions apparaît alors.
	\item Pour corriger toutes les occurrences de ce mot dans le document édité, on utilise la commande \og{}:spellr\fg{}.
	\item Pour atteindre la prochaine faute, on tape \og{}]s\fg{}. Pour effectuer la même recherche vers le haut, on utilise \og [s \fg.
	\item Pour désactiver et réactiver la correction orthographique, il faut utiliser les commandes \og{}:set spell\fg{} et \og{}:set nospell\fg{}.
\end{itemize}

%%%%%%%%%%%%%%%%%%%%%%%%%%%%%%%%%%%%%%%%%%%%%%%%%%

\section{Personnaliser le dictionnaire}
Un dictionnaire personnalisé permet d'enrichir le correcteur en lui indiquant les mots qu'il ne doit pas considérer comme des fautes (les noms et les prénoms par exemple).

On peut ajouter un mot en plaçant le curseur dessus et en tapant \og{}zg\fg{}. Pour annuler cette opération, on utilise \og{}zug\fg{}.

%%%%%%%%%%%%%%%%%%%%%%%%%%%%%%%%%%%%%%%%%%%%%%%%%%

\section{Pour conclure}
La correction orthographique est très efficace dans Vim. Elle a entre autre l'avantage de tenir compte des spécificités du type de fichier édité\footnote{Sous condition d'avoir activé la reconnaissance syntaxique avec la commande \og{}:syntax~on\fg{}}. Par exemple, quand on active la correction orthographique sur un fichier .c ou .java, Vim ne va corriger que les déclarations de chaînes de caractères entre guillemets et les commentaires. De même, lorsque l'on travaille avec un fichier .tex, il ne va pas tenir compte du balisage et des formules.

Malheureusement et comme souvent avec les éditeurs libres la grammaire n'est pas prise en compte.

%\section*{Licence}
~\\

\begin{center}
    \href{http://creativecommons.org/licenses/by-sa/2.0/fr/}{\includegraphics[width=.15\linewidth]{images/cc_by_sa_small}}\\
	\small{Creative Commons BY-SA}
\end{center}

\end{document}
